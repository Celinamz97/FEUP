\documentclass[10pt,landscape,a4paper]{article}

\usepackage[top=6.35mm,bottom=6.35mm,left=6.35mm,right=6.35mm]{geometry}

\usepackage{etoolbox}

\usepackage{amsmath, mathtools}

\usepackage[english]{babel}
\usepackage{microtype}
\usepackage{unicode-math}
\setmainfont{TeX Gyre Pagella}
\setmathfont{TeX Gyre Pagella Math}

\usepackage{xcolor}

\usepackage{multicol}

\usepackage{enumitem}
\setlist[itemize]{leftmargin=*}

\usepackage{nth}
\usepackage[binary-units]{siunitx}

\makeatletter
\renewcommand{\section}{\@startsection{section}{1}{0mm}{.2ex plus 1pt}{.2ex}{\bfseries\color{section}}}
\renewcommand{\subsection}{\@startsection{subsection}{2}{0mm}{.2ex}{.2ex}{\bfseries}}
\renewcommand{\subsubsection}{\@startsection{subsubsection}{3}{0mm}{.2ex}{.2ex}{\footnotesize\bfseries}}
\renewcommand{\paragraph}{\@startsection{paragraph}{4}{0mm}{.2ex}{.1ex}{\footnotesize\bfseries}}

\g@addto@macro \small {%
  \setlength\abovedisplayskip{0pt plus 0pt minus 0pt}%
  \setlength\belowdisplayskip{0pt plus 0pt minus 0pt}%
  \setlength{\abovedisplayshortskip}{0pt plus 0pt minus 0pt}%
  \setlength{\belowdisplayshortskip}{0pt plus 0pt minus 0pt}%
}

\flushbottom
\setlength{\maxdepth}{0pt}
\setlength{\parindent}{0pt}
\renewcommand{\baselinestretch}{.8}
\pagestyle{empty}
\AtBeginDocument{\small}

% NOTE(nox): Math color
\everymath{\color{math}}
\apptocmd{\NCC@startdisplay}{\color{math}}{}{
  % NOTE(nox): This is the error section -> if nccmath is not included
  \apptocmd{\mathdisplay}{\color{math}}{}{}

  % NOTE(nox): Remove space between section & math
  \pretocmd{\mathdisplay}{\noindent}{}{}
  \pretocmd{\start@gather}{\noindent}{}{}
  \pretocmd{\start@align}{\noindent}{}{}
  \pretocmd{\start@multline}{\noindent}{}{}
}
\makeatother

% ==========================================================================================
\definecolor{section}{cmyk}{1,.72,0,.38}
\definecolor{math}{cmyk}{1,.72,0,.38}
\definecolor{highlight}{cmyk}{0,0.3,0.5,0.41}

\renewcommand\emph[1]{\textcolor{highlight}{#1}}

\begin{document}
\begin{multicols*}{4}
  \section{Physical layer}
  \begin{itemize}
  \item Unreliable transmission and reception of digital data
  \item Virtual bit stream
  \end{itemize}

  \subsection{Channel capacity}
  $R$ is \si{\bit/\second}, $M$ is number of levels, $B$ is bandwidth:
  \begin{description}
  \item[Hartley] $R = 2B \log_2 M$
  \item[Shannon] $R = B \log_2 (1 + \text{SNR})$
  \end{description}

  \section{Data link layer}
  \begin{itemize}
  \item Framing
  \item Deals with errors (eg. discards)
  \item Regulates data flow
  \end{itemize}

  \subsection{Error probabilities}
  $p$ is BER, $\ell$ is frame length in \si{\bit}:
  \begin{align*}
    P(\text{has errors}) &= 1 - (1-p)^\ell = \text{FER} \\
    P(\text{has } i \text{ errors}) &= {\ell\choose i} p^i(1-p)^{\ell-i}
  \end{align*}

  \subsection{ARQ}
  Automatic retransmission of missing packets or packets with errors.
  $p_e \equiv \text{FER}$

  \begin{gather*}
    t _\text{prop} = \frac{d}{v} \qquad
    t _\text{trans} = \frac{L}{R} \\
    a = \frac{t _\text{prop}}{t _\text{trans}} = \frac{t _\text{prop} R}{L} \quad
    \max_\text{throughput} = R S
  \end{gather*}

  \subsubsection{Stop-and-Wait}
  \[ W = 1 \Rightarrow S = \frac{1 - p_e}{1 + 2a} \]

  Sender:
  \begin{enumerate}
  \item Sends frame
  \item Waits for acknowledgement
    \begin{itemize}
    \item If ACK received, send new frame
    \item If NACK received or timeout, resend this frame
    \end{itemize}
  \end{enumerate}

  Receiver:
  \begin{enumerate}
  \item Receives frame
    \begin{itemize}
    \item No errors $\Rightarrow$ send ACK
    \item Has errors $\Rightarrow$ send NACK
    \end{itemize}
  \end{enumerate}

  \subsubsection{Sliding window}
  \begin{itemize}
  \item Using $n$ bits in each packet, we can use $M = 2^n$ sequence numbers.
  \item There may be $W (< M)$ unacknowledged packets in traffic

  \item The efficiency values are upper bounds, as:
    \begin{itemize}
    \item The acknowledgement packet is considered having zero length ($\Rightarrow$ zero transmission delay)
    \item Ignores processing and queuing delays
    \end{itemize}
  \end{itemize}

  \paragraph{Go-Back-N}
  \begin{align*}
    W &\le M - 1 = 2^n - 1 \\
    S &= \frac{W}{1 + 2a},\, W < 1+2a \\
    S &= \frac{W (1 - p_e)}{(1 + 2a)(1 - p_e + Wp_e)},\, W < 1+2a \\
    S &= \frac{1 - p_e}{1 + 2ap_e},\, W \ge 1+2a
  \end{align*}

  \begin{itemize}
  \item Send packets acknowledging correctly received ones, informing we are waiting for sequence number $k$ (next frame!); so, it is cumulative

  \item Receiver
    \begin{itemize}
    \item \emph{Discards} frames with \emph{errors}
    \item When \emph{out of sequence}, send \texttt{REJ(k)} (still expecting k) for the \nth{1} time, simply discard the following times
    \end{itemize}

  \item When transmitter receives \texttt{REJ(k)}, retransmits $k, k+1, \dotsc$
  \end{itemize}

  Possible extensions are piggy-backing (waiting \& joining several ACKs into a single cumulative one), and sending the ACKs in packets going in the reverse direction.

  \paragraph{Selective Reject}
  \begin{align*}
    W &\le \frac{M}{2} = 2 ^{n-1} \\
    S &= \frac{W (1 - p_e)}{1 + 2a},\, W < 1+2a \\
    S &= 1 - p_e,\, W \ge 1+2a
  \end{align*}

  Like GBN, but stores out of order, and also sends \texttt{REJ(k)} when damaged.
  After receiving \texttt{REJ(k)}, only resend $k$.

  \subsection{End to end capacity}
  $R$ is link capacity, $k$ is number of links:
  \begin{description}
  \item[LL ARQ] $C = R (1 - \text{FER}) \to$ more complex
  \item[E2E ARQ] $C = R (1 - \text{FER})^k$
  \end{description}

  \section{Delay models}
  Queue analysis is always done in \emph{steady state}, not instantaneously!

  \begin{description}
  \item[{\(C\)}] Channel capacity \emph{after} the queue
  \item[{\(N\)}] Average number of packets in the \emph{system}
  \item[{\(T\)}] Average delay experienced by packets (\(T_w\) is queue time, \(T_s = \frac{1}{\mu}\) is service time)
  \item[{\(\lambda = \frac{R}{L}\)}] Arrival rate of packets
  \item[{\(\mu = \frac{C}{L}\)}] Service rate of packets
  \item[{\(\rho = \frac{\lambda}{\mu}\)}] Traffic intensity (occupation)
  \item[{Little's Theorem}] \(N = \lambda T \Rightarrow N_w = \lambda T_w\)
  \end{description}

  \subsection{M/M/1 queue}
  Poisson arrival, exponential service time, based on Markov chains, 1 server, infinite queue length.
  \begin{align*}
    N &= \frac{\rho}{1 - \rho} = \frac{\lambda}{\mu - \lambda} \qquad \text{as } \rho \to 1 \Rightarrow N \to \infty \\
    T &= \frac{1}{\mu - \lambda} \\
    \Rightarrow T_w &= T - T_s = \frac{1}{\mu - \lambda} - \frac{1}{\mu} = \frac{\rho}{\mu (1 - \rho)} \\
    N_w &= \lambda T_w = N - \rho
  \end{align*}

  When averaging things between queues, use an \emph{weighted average}!

  \subsubsection{M/M/1/B queue}
  Queue length is $B \Rightarrow$ packets are lost when queue is full, with probability $P(B)$.
  \[
    P(n) = \frac{\rho^n(1 - \rho)}{1 - \rho^{B + 1}} \Rightarrow P(B) =
    \begin{cases*}
      \frac{1}{B+1}, &$\rho = 1$  \\
      \sim\frac{\rho - 1}{\rho}, &$\rho \gg 1$
    \end{cases*}
  \]

  \subsection{M/D/1 queue}
  Constant packet length!
  \begin{align*}
    N &= \rho + \frac{\rho^2}{2 (1 - \rho)} & N_w &= \frac{\rho^2}{2 (1 - \rho)} \\
    T &= \frac{1}{\mu} + \frac{\rho}{2\mu (1 - \rho)} & T_w &= \frac{\rho}{2 \mu (1 - \rho)}
  \end{align*}

  \section{Medium Access Control}
  \emph{Ideal}: Coordinate stations to share medium.
  Decentralized, without coordination, simple.

  \subsection{Aloha}
  Transmits; if no ACK is received, delays retransmission randomly.
  In slotted, only may send at the beginning of slots (less collisions!).

  \subsection{Carrier Sense Multiple Access - CSMA}
  Listen before transmitting, defer if busy.
  Vulnerable time is $t_\text{prop}$.
  A station can only be sure there is no collision after $2 t_\text{prop}$.
  Slots have $2 t_\text{prop}$.

  \begin{description}
  \item[non-persistent] wait random interval and retry
  \item[persistent] transmit as soon medium is free
  \item[p-persistent] transmit with probability $p$ on next \emph{free} slot, else defer
  \end{description}

  \subsubsection{CSMA/CD - Collision Detection}
  Listen while transmitting.
  If collision detected, stop and retransmit after delay, following \emph{binary exponential backoff}.
  After $i$th consecutive collision, attempt to transmit after a random number of slots, uniformly distributed between $0$ and $2^i - 1$.
  Minimum frame size \emph{required}.
  No ACKs needed.

  \subsubsection{CSMA/CA - Collision Avoidance}
  When there is a packet to transmit, transmit after DIFS idle, if free.
  If busy, wait random backoff interval; decrease after idle > DIFS.
  Stop timer when busy.
  Consecutive packets are delayed to prevent channel capture.

  \emph{ACKs are required}, because collision may not be visible.

  \subsection{Ethernet MAC}
  $\SI{48}{\bit} = \SI{6}{\byte}$ address.
  \begin{description}
  \item[Single coaxial] Fragile, single collision domain
  \item[Repeater hubs] Repeat bits to all \emph{other} ports, single collision domain
  \item[Bridge]
    \begin{itemize}
    \item Forward only to single port (based on MAC address)
    \item Separates collision domains
    \item Transparent to hosts
    \item Self-learning: records address $\to$ port of received packet, looks up destination MAC, if found, send to destination (if \emph{another} port);
      Else, floods to all other ports
    \end{itemize}
  \end{description}

  Only thing remaining from the original is the \emph{frame format} \& \emph{MAC addresses}.
  Now, it is point-to-point and uses mostly Unshielded Twisted Pair (UTP) \textemdash{} cheaper, easier to install, full-duplex.

  \subsection{VLANs}
  Separate \emph{broadcast} domains, like routers.

  \section{Network layer}
  \begin{description}
  \item[Forwarding] Data plane, take packets from input ports to output ports
  \item[Routing] Control plane, compute forwarding table entries, distributed
  \end{description}

  \subsection{Virtual circuit network}
  Connection-oriented service.
  All data goes through same pre-defined path.
  \begin{enumerate}
  \item Establishment
  \item Data transfer
  \item Circuit termination
  \end{enumerate}

  Packets only need a VC identifier, that \emph{changes with each} link.
  Each router maintains state for each VC, allocating resources for it.
  One router fails, all circuits through that are \emph{terminated}.

  May have \emph{unused bandwidth}.
  Guaranteed QoS.

  \subsection{Datagram networks}
  No connection.
  Each router decides, each time, where a packet should go based on destination address.
  May go through \emph{different paths}!
  Resilient to router crashes.

  \subsection{Routers}
  \subsubsection{Input port}
  \begin{itemize}
  \item Look up output port
  \item Send to switching fabric
  \item Queuing if needed
    \begin{itemize}
    \item Line rate > forwarding rate
    \item \emph{Head of line blocking}
    \end{itemize}
  \end{itemize}

  \subsubsection{Switching fabric}
  \begin{description}
  \item[Memory switch] Switching controlled by CPU, packets go through bus twice
  \item[Shared bus] CPU not involved, bus contention
  \item[Crossbar switch] Simultaneous forwarding, $N^2$ interconnections.
  \end{description}

  \subsubsection{Output port}
  Scheduling and queuing (arrival rate > line rate)

  \subsection{Internet Protocol}
  \subsubsection{Fragmentation \& reassembly}
  \begin{description}
  \item[{MTU}] Maximum Transfer Unit, may be different at each link
    \begin{itemize}
    \item Large IP datagrams may be fragmented
    \item Reassembled at the \emph{final destination}
    \end{itemize}
  \end{description}

  \subsubsection{Addressing}
  \SI{32}{\bit} addresses, each network interface has 1

  \subsubsection{Subnets}
  Same broadcast domain $\Rightarrow$ can reach directly

  Forwarding table contains pairs of (Network address+mask $\to$ output port).
  Use \emph{longest prefix match} to know where to send.

  There are $2^{32 - \text{mask size}} - 2$ usable addresses.

  \paragraph{Special IP addresses}
  \begin{description}
  \item[{All zeros}] This host
  \item[{Zeros + Host}] Host on this network
  \item[{All ones}] Broadcast (on local?)
  \item[{Network + ones}] Broadcast on specific network
  \item[{Network + zeros}] Network address
  \end{description}

  \subsubsection{Address Resolution Protocol (ARP)}
  Resolves IP addresses into MAC addresses.

  \subsection{ICANN}
  Allocates addresses and manages DNS.

  \subsection{Network Address Translation (NAT)}
  All packets \textbf{leaving} a NAT have the same IP address.
  Communications identified by \emph{port numbers}.

  \subsubsection{NAT traversal}
  An external client can't directly contact an internal host!
  $\Rightarrow$ port forwarding is \textbf{one} solution

  \subsection{Internet Control Message Protocol (ICMP)}
  Used for \emph{Layer 3 error or control messages}.
  Carried in IP packets.

  May be used for pings (echo request \& reply), traceroute (TTL expired), or ICMP redirects (which is sent by the routers in order to indicate a more efficient route).

  \subsection{IPv6}
  IPv4 addresses are depleting!
  We may use NATs, reclamation of unused space, ISP-level NATs, IPv4 buying/selling to mitigate.
  Or switch.

  \section{Routing}
  A tree is a \emph{connected} graph with \emph{no cycles}.
  Cost: $C_\text{total}(T) = \sum w(e_i')$

  A \emph{spanning} tree is built from a graph and goes to all its nodes.
  In a \emph{minimum} spanning tree (there is only one for each graph) \[C_\text{total}(T^*) = \min_T(C_\text{total}(T))\]

  \subsection{Link-state routing}
  Every router have \emph{full view} of the entire state of the network, all of them broadcast their local states.
  $\Rightarrow$ not scalable!

  Periodic beacons are used to detect \emph{topology changes}.
  When changed, they \emph{flood} their state (also periodically).

  \subsection{Distance vector algorithms}
  Each router has:
  \begin{itemize}
  \item Has cost of direct links $c(x, v)$
  \item Estimate of total cost to all other nodes $D_{x}(y)$
  \item Estimate of \emph{their neighbors} to all other nodes!
  \end{itemize}
  The estimate is $D_{x}(y) = \min(c(x, v) + D_{v}(y))$

  Each node recomputes estimation and \emph{notifies} when new changes arrive.

  \subsection{Spanning Tree Protocol}
  Used by \emph{switches} to build a \emph{loop-free} topology.
  Root is the switch with \emph{smallest ID}.
  Messages are \texttt{(RootNode, DistanceFromRoot, ThisNode)}.

  \subsection{Flow network model}
  Comm. networks are \emph{queue} networks!
  This is used to estimate limits.

  \emph{Maximum capacity} Maximum amount of flow between source and sink, with the minimum amount of cuts to separate them.

  \section{Transport layer}
  Provides end-to-end service.

  \subsection{Why are there separate network and transport layers?}
  Users have \emph{no real control over the network}.
  They can improve the situation by using \emph{another layer} that improves the quality of the service.
  Transport primitives can be implemented as calls to library procedures.

  \subsection{Multiplexing}
  Communications identified by \texttt{(Source IP, Port - Destination IP, Port)} and transport type.

  \subsection{User Datagram Protocol - UDP}
  Connectionless, no reordering, no ARQ.

  $\Rightarrow$ Small protocol overhead.

  \subsection{Transmission Control Protocol - TCP}
  Connection oriented, has \emph{flow control}, has \emph{congestion control}.

  \subsubsection{Sequence number}
  Bytes are numbered sequentially, sequence number is the first byte of the packet.
  Stream is divided in segments, at most of \emph{Maximum Segment Size} (\texttt{MSS}).

  \subsubsection{ARQ}
  Variation of Go-Back-N, retransmits of a \emph{single packet} at a time.

  May have selective ACK for out-of-order, retransmits on timeout or 3 out of order packets.

  \subsubsection{Adaptive retransmission}
  Measures $\mathrm{RTT}$ for each segment/ACK (except in retransmissions!).
  Rolling average for the $\mathrm{RTT}$.
  Timeout is $2\mathrm{RTT}$.

  \subsubsection{Flow control}
  Doesn't overflow receiver with data!
  Uses sliding windows for that.

  \subsubsection{Congestion control}
  Regulates traffic based on ACKs, \texttt{Congestion\-Window} is continuously adapted.

  \paragraph{Congestion control methods}
  \begin{enumerate}
  \item Slow start
    \begin{itemize}
    \item \emph{Double} \texttt{CongestionWindow} for \textbf{each RTT} (+1 for each ACK)
    \item On timeout
      \begin{itemize}
      \item Set \texttt{threshold} to half the \texttt{CongestionWindow}
      \item Restart slow start
      \item Go to congestion avoidance phase when \texttt{threshold} is hit
      \end{itemize}
    \end{itemize}

  \item Additive increase and multiplicative decrease
    \begin{itemize}
    \item \emph{Increment} \texttt{CongestionWindow} for \textbf{each RTT}
    \item When packet is lost (3 repeated ACKs):
      \begin{itemize}
      \item \texttt{CongestionWindow} is divided by 2
      \item Retransmit packet
      \end{itemize}
    \end{itemize}

  \item Fast retransmission
    \begin{itemize}
    \item 3 repeated ACKs? Retransmit!
    \end{itemize}
  \end{enumerate}
\end{multicols*}
\end{document}
